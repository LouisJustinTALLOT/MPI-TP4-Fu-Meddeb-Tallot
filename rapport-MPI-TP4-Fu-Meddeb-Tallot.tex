%%%%%%%%%%%%%%%%%%%%%%%%%%%%%%%%%%%%%%%%%%%%%%%%%%%%%%%%%%%%%%%%%%%%%%%%%%%%%%%%%%%%%%
% TEMPLATE FOR PHYS250 WORKSHEET
% This template uses the Memoir class. It is a very powerful class for 
% creating documents such
% as reports, papers and theses. You can find more information at CTAN, the Comprehensive
% TeX Archive Network. These is a long manual that describes how to use Memoir.
% https://www.ctan.org/pkg/memoir?lang=en

%%%%%%%%%%%%%%%%%%%%%%%%%%%%%%%%%%%%%%%%%%%%%%%%%%%%%%%%%%%%%%%%%%%%%%%%%%%%%%%%%%%%%

\title{Matériaux pour l'Ingénieur\vspace{0.5em}\\TP n°4\vspace{0.5em}\\Compte-rendu}
\author{Jia \textsc{Fu}, Paul \textsc{Meddeb}, Louis-Justin \textsc{Tallot}}
\date{Mardi 15 juin 2021}


\documentclass[12pt,oneside,oldfontcommands]{memoir}



%	MARGIN AND HEADER/FOOTER SIZES

\setlrmarginsandblock{2.5cm}{2.5cm}{*}  		% left/right margins
\setulmarginsandblock{2.5cm}{2.5cm}{*} 			% top/bottom margins
\checkandfixthelayout							% checks the layout is correct
\setlength{\parindent}{0in}  					% no indent on start of paragraph
							


%  PACKAGES

\usepackage{amsmath,amsthm,amssymb,amsfonts}			% math fonts
\usepackage[french]{babel}								% hyphenation rules for english
\usepackage[T1]{fontenc} 
\usepackage{graphicx}
% for importing pdf files 
\usepackage{siunitx}									% si units - extremely useful
% defines the dvips color names
\usepackage[usenames,dvipsnames,svgnames,table]{xcolor}	
\usepackage{color,soul} 	% for highlight hi - hyphenation, underlining
\setulcolor{red} 										% set underline color
\setstcolor{green} 										% set overstriking color
\sethlcolor{green} 										% set highlighting color
\usepackage{tabularx}
\usepackage{float} 

\usepackage{pgfplots}
\DeclareUnicodeCharacter{2212}{−}
\usepgfplotslibrary{groupplots,dateplot}
\usetikzlibrary{patterns,shapes.arrows}
\pgfplotsset{compat=newest}

\usepackage{fancyhdr}
\pagestyle{fancy}
\fancyhf{}
\rhead{École des Mines de Paris}
\lhead{MPI - TP n°4 - Compte-rendu}
\rfoot{\vspace{-0.7 cm}\thepage}
\lfoot{\includegraphics[width = 3 cm]{logo_mines.png}}


%  GRAPHICS PATH

\graphicspath{{images/}{./}{images_optique/}{images_meb/}}		% put your figures in a folder called images



%  SOME NEW FUNCTIONS FOR IMPORTING FIGURES

\newcommand{\placefigure}[1]{\centerline{\includegraphics[width=2 in]{#1}}} 
\newcommand{\placefigureandscale}[2]{\centerline{\includegraphics[width=#2 in]{#1}}} 



%-------------------------------------------------------------------------------------
%	TITLE PAGE MACRO
%------------------------------------------------------------------------------------
\makeatletter
\def\maketitle{%
  \null
  \thispagestyle{empty}
  \begin{center}\leavevmode
       \vskip 2cm
       \normalfont
       \includegraphics[width=0.5\columnwidth]{logo_mines.png}
       \vskip 1cm
    %   \textsc{\Large PHYS250 Worksheet}\\[0.5 cm]
	     {\large Mardi 15 juin 2021\par}
       \vskip 1.0cm
	\rule{\linewidth}{0.2 mm} \\[0.4 cm]
	{ \huge \bfseries \@title}\\
	\rule{\linewidth}{0.2 mm} \\[1.5 cm]
	
	\vspace{1cm}
	
	\Large{\emph{Étudiants} : Jia \textsc{Fu}, Paul \textsc{Meddeb}, Louis-Justin \textsc{Tallot}}
	\vspace{0.5cm}
	\Large{\emph{Encadrants} : Robin \textsc{Mallick}, Mohamed \textsc{Shokeir}}
	
% 	\begin{minipage}{0.5\textwidth}
% 		\begin{flushleft} \large
% 			\emph{Name:} \studentone\\
% 			Student Number: \studentonenumber
% 			\end{flushleft}
% 			\end{minipage}~
% 			\begin{minipage}{0.4\textwidth}
% 			\begin{flushleft} \large
% 			\emph{Partner:} \studenttwo\\
% 			Student Number: \studenttwonumber
% 		\end{flushleft}
% 	\end{minipage}\\[2 cm]
   \end{center}
   \vfill
   \null
   \cleardoublepage
  }
\makeatother


%	START OF DOCUMENT


\begin{document}
%\large 
\maketitle
\frontmatter
\let\cleardoublepage\clearpage
\mainmatter
\sloppy



\counterwithout{figure}{chapter}
\counterwithout{table}{chapter}

\section*{Introduction générale}
% texte pour l'introduction générale

Les réacteurs d'avion sont des assemblages complexes de pièces soumises à de
très fortes contraintes. En effet, l'air qui est aspiré à la température 
ambiante, est ensuite compressé



\section*{Introduction scientifique}
L'aube de turbine évolue dans un environnement particulièrement oxydant et corrosif. 
En particulier la haute température (\SI{1250}{\celsius}) et les contraintes auxquelles est soumise 
l'aube sont propices au fluage, c'est à dire la déformation lente de la pièce soumise à une contrainte constante sous une forte température. Il convient donc d'étudier l'influence des 
caractéristiques micro-structurales sur le fluage, dont la diffusion est le principal 
vecteur puisque c'est sous son effet que les atomes migrent en provoquant l'allongement du matériau.\\


La première forte évolution technologique dans le domaine a été le passage à la conception 
d'aube mono-cristallines vers la fin des années 1970. En effet les joints de grain sont des
"autoroutes à diffusion" et le fait de n'avoir qu'un seul grain améliore grandement les 
performances en fluage. Ainsi, les aubes sont aujourd'hui réalisées en super-alliage de 
nickel à base d'aluminium par refroidissement et croissance d'un unique grain initial.\\


D'autre part, les défauts micro-structurels ont également un impact sur la résistance 
mécanique du matériau. Par exemple, la trop forte présence de pores pose problème, 
ainsi que les hétérogénéités chimiques. Lors de l'étape de fonderie, le refroidissement
n'est pas homogène et certaines zones se solidifient avant d'autres ce qui fait 
que le liquide est appauvri de certains éléments : on parle alors de dendrites. 
C'est ce contraste chimique résultant entre dendrites et zones inter-dendritiques
qui diminue les capacités de résistance mécanique de la structure.\\


À l'échelle des phases, un agencement ordonnée des phases $\gamma'$ dans la 
matrice $\gamma$ permet de moins laisser se propager les champs de contraintes de 
certains défauts tel que la dislocation.\\


Deux questions se posent : Quelles sont les caractéristiques micro-structurelles 
optimales du point de vue de la résistance du matériau au fluage et aux autres 
contraintes de son milieu ? Et dans quelle mesure différents traitements thermiques 
permettent-ils d'atteindre cet optimum ?\\


La première question est traitée par les industriels qui ont déterminé différentes 
caractéristiques reconnues comme optimales (fraction volumique de phase $\gamma'$, 
taille des précipités de phase $\gamma'$, différence des paramètres de maille 
entre $\gamma$ et $\gamma'$...).\\


Nous nous concentrons donc dans ce TP sur l'influence du traitement thermique en observant l'organisation et les défauts de la micro-structure après chaque étape de ce traitement.\\



\section*{Objectifs}
Pour pouvoir obtenir la microstructure composée des phases $\gamma$ et $\gamma'$ avec des taille et fraction de précipités optimales, les superalliages à base de Nickel sont soumis à une gamme de traitements thermiques : remise en solution puis trempe, premier revenu et deuxième revenu. L’objectif de la remise en solution (à des températures correspondantes au domaine monophasé $\gamma$) est de faire disparaître les défauts qui sont apparus pendant le procédé de fonderie : ségrégations chimiques et zones d’agrégats eutectiques ($\gamma / \gamma'$). Pendant la trempe (pour revenir à une température correspondant au domaine $\gamma + \gamma'$), la microstructure issue de la remise en solution est figée et il n’y aurait pas de transformation de phase. Les premier et deuxième revenus (à des températures correspondantes au domaine biphasé $\gamma / \gamma'$) conduisent à la précipitation et croissance de la phase $\gamma'$ dans la matrice $\gamma$. Les propriétés finales des superalliages à base de Nickel dépendent de la microstructure finale en termes de taille et de fraction de précipités $\gamma'$ ainsi que de la fraction de défauts issues du procédé de fonderie qui n’ont pas disparu pendant les traitements thermiques.
L’objectif de ce travail pratique est d’étudier l’influence des traitements thermiques sur la microstructure et sur les défauts formés pendant le procédé de fonderie : pores, zones d’agrégats eutectiques et hétérogénéités chimiques entre dendrites et zones inter-dendritiques. Pour cela, des observations de microstructures couplées avec de l’analyse d’image seront réalisées.


\section*{Démarche mise en œuvre}
1. Préparation métallographique des échantillons : polissage.\\
2. Observations des microstructures au microscope optique.\\
3. Attaque chimique à l’eau régale.\\
4. Observation des microstructures au microscope électronique à balayage (MEB)\\
5. Analyses d’images : à partir des images obtenues au microscope optique (au moins 5 champs d’observation par échantillon), déterminer les fractions de défauts observés (agrégats
eutectiques ($\gamma / \gamma'$), pores et autres).\\
6. Etude de la ségrégation chimique des éléments en fonction du traitement thermique par
profils de composition chimique par analyse dispersive en énergie (EDS).\\
7. Analyse des résultats :\\
a. Conclusion sur les types de défauts observés à l’état brut de fonderie. Donner une explication quant à l’apparition de ces défauts.\\
b. Tracer les courbes d’évolution des fractions de défauts en fonction des traitements thermiques reçus.\\
c. Conclure quant à l’influence des traitements thermiques sur la fraction de défauts et sur la microstructure $\gamma / \gamma'$. \\
\newpage

\section*{Description des outils utilisés}
\centerline{\includegraphics[width=0.55\textwidth]{images/WechatIMG1126.jpeg}}

\\\\
L'étape de polissage et nettoyage est cruciale pour ne pas avoir 
trop de défauts à la surface de l'échantillon:
on polit la pièce sur un polisseur tournant (1200 -> 2400 (eau)), puis on la passe au bain à ultrasons 2 minutes pour la nettoyer.
Ensuite on fait le passage à 2µm et 1µm (lubrifiant),
puis on la repasse au bain à ultrasons.\\
\\
\\
\centerline{\includegraphics[width=0.35\textwidth]{images/WechatIMG1128.jpeg}}
\\
On utilise un mélange composé de deux tiers d'acide chlorhydrique 
et d'un tiers d'acide nitrique.
On attend 2 minutes que les deux acides réagissent ensemble.
On observe que la solution passe de transparente à jaune-orangée.
Puis on fait tremper les pièces 45 secondes chacune dans la solution.
\\

\centerline{\includegraphics[width=0.35\textwidth]{images/optique.jpg}}
\\
Le microscope optique permet d'observer nos échantillons à l'échelle de la centaine de $\mu m$ ce qui permet de rapidement repérer les défauts les plus gros tels que les pores ou les fissures.
\\

\centerline{\includegraphics[width=0.4\textwidth]{images/JEOL_JSM-6340F.jpg}}
\\
Des observations au MEB (microscope électronique à balayage) sont réalisées après refroidissement sous haute contrainte. Le microscope optique ne nous permettait pas d'observer l'influence du traitement thermique sur l'agencement des phases $\gamma$ et de $\gamma'$. La résolution du MEB de l'ordre du $\mu m$ nous permet d'observer la phase $\gamma'$ dans la matrice $\gamma$ ainsi que les éventuels agrégats eutectiques. La détection d'électrons rétro-diffusés permet de plus de voir à l'image les phases $\gamma$ et $\gamma'$. En effet les atomes légers de la phase $\gamma'$ sont moins déviés que les atomes plus lourds de la phase $\gamma$. Ainsi les couloirs de matrice $\gamma$ apparaissent en blanc et la phase $\gamma'$ apparaît en noir.
\newpage



\section*{Présentation des résultats obtenus}
Nous avons ainsi étudié trois types d'échantillons différents :\\
\begin{itemize}
    \item Le brut, monocristallin, en sortie de fonderie
    \item L'alliage après remise en solution (premier traitement thermique)
    \item L'alliage en fin de traitement, après les deux revenus.\\\\
\end{itemize}



Nous devons donc étudier leur microstructure pour déterminer l'influence des 
différents traitements thermiques sur l'alliage, étant donné que cette microstructure
est le paramètre clé qui va déterminer les propriétés mécaniques de l'alliage,
et notamment sa résistance au fluage.

\subsection*{Echantillon brut de fonderie}

Étudions tout d'abord les propriétés du brut après fonderie.\\\\


% \centerline{\includegraphics[width=0.75\textwidth]{images_optique/brut.pdf}}
% \legend{Le brut de fonderie vu au microscope optique}

\begin{figure}[htbp]
    \centering
    \includegraphics[width=0.75\textwidth]{images_optique/brut.pdf}
    \caption{Echantillon brut de fonderie vu au microscope optique}
    \label{}
\end{figure}


Vu au microscope optique, sa surface (après polissage) est lisse
et présente peu de défauts. Cependant, la surface possède 
également de petites tâches noires : ces tâches sont des \emph{pores},
c'est à dire des trous à la surface de l'échantillon. Il y a à ces endroits 
des lacunes importantes dans la maille cristalline, et il manque donc 
une petite portion du matériau. Nous observons également de plus 
petites tâches grises : il s'agit d'agrégats eutectiques, formés lors du refroidissement du liquide.


% \centerline{\includegraphics[width=0.75\textwidth]{images_optique/brut2.pdf}}
% \legend{Le brut de fonderie vu au microscope optique}

\begin{figure}[htbp]
    \centering
    \includegraphics[width=0.75\textwidth]{images_optique/brut2.pdf}
    \caption{Echantillon brut de fonderie vu au microscope optique}
    \label{<label>}
\end{figure}

On mesure la proportion de défauts visibles au microscope optique à l'aide du 
logiciel d'analyse d'images ImageJ. Nous obtenons les résultats suivants : \\
\begin{table}
    \centering
    \caption{Proportion des différents défauts dans l'échantillon brut de fonderie}
    \begin{tabular}{c|c}
        \textbf{Type de défaut}  & \textbf{Proportion observée}  \\
        \hline
        Pore               & 0,340 \% \\
        Agrégat eutectique & 0,252 \% \\
    \end{tabular}

\end{table}

Nous voyons ainsi une proportion assez nette de défauts au sortir de la fonderie. 
Les agrégats eutectiques apparaissent pendant le refroidissement 
et la solidification du liquide. En effet, la vitesse de refroidissement dans 
la pièce est très difficile à contrôler précisément, et est donc nécessairement
inhomogène. Ainsi, certaines zones vont commencer à solidifier avant d'autres, et 
vont donc modifier la composition chimique du liquide qui les entoure. 

% \centerline{\includegraphics[width=0.75\textwidth]{images/diagramme_phase.png}}
% \legend{Diagramme Ni-Al avec les structures cristallines de la phase $\gamma$ et de la phase $\gamma'$}

\begin{figure}[htbp]
    \centering
    \includegraphics[width=0.75\textwidth]{images/diagramme_phase.png}
    \caption{Diagramme Ni-Al avec les structures cristallines des phases $\gamma$ et $\gamma'$}
    \label{<label>}
\end{figure}

% Quand on regarde le diagramme de phases nickel-aluminium, nous voyons que 
% lors de la solidification, le liquide va s'appauvrir en nickel et sa composition
% va tendre vers celle du liquide eutectique. Cela conduit donc à la formation de zones
% donc la composition est celle de l'assemblage $\gamma$/$\gamma'$, et d'autres zones 
% contenant uniquement la phase $\gamma'$ qui correspondent aux agrégats eutectiques.

De plus, les dendrites, qui sont les parties qui solidifient en premier, sont composées 
principalement de phase $\gamma$ (d'après le diagramme de phases nickel-aluminium), 
ce qui va modifier l'équilibre dans le liquide.
Les espaces inter-dendritiques vont ensuite se former,  plus riches en phase $\gamma'$, 
suivis enfin par les agrégats eutectiques composés de phase $\gamma'$.

Quant aux pores, leur apparition est liée à un flux trop faible de soluté 
dans les espaces inter-dendritiques, c'est-à-dire que la solidification de 
certaines zones se fait sans que le liquide ou la diffusion aient eu le 
temps de combler certains vides créés par la contraction du liquide qui
se refroidit. Nous avons donc un certain nombre de lacunes et de pores qui
apparaît ; ce nombre varie en fonction de la vitesse de refroidissement 
de la pièce.


\subsection*{Echantillon remis en solution}

Regardons maintenant notre échantillon après la remise en solution. 
Ce traitement consiste à réchauffer la pièce à très haute température (\SI{1300}{\celsius}),
mais en-dessous du point de fusion de l'alliage : aucun liquide n'est donc formé. 

Ce traitement a pour but d'homogénéiser la composition de l'alliage. En effet, nous avons 
vu différentes zones se solidifient à différents moments, ce qui place une partie de l'alliage
hors équilibre (comme par exemple les agrégats eutectiques). Le chauffage intense va également 
permettre d'accèlérer la diffusion dans l'alliage, et donc faciliter le mouvement des atomes 
(notamment les plus lourds comme le rhénium). 

Observons donc notre échantillon ayant subit une étape de remise en solution au microscope optique.

\begin{figure}[htbp]
    \centering
    \includegraphics[width=0.55\textwidth]{images_optique/res.pdf}
    \caption{Echantillon après traitement de remise en solution}
    \label{<label>}
\end{figure}

Nous voyons que les agrégats eutectiques, très visibles sur le brut, ont quasiment disparu.
Ils étaient en effet nettement hors équilibre et donc peu stables thermodynamiquement. 

Néanmoins, nous avons augmenté le nombre de pores. Cela est lié au fait que, la température 
augmentant, nous avons également augmenté la mobilité de tous les atomes et donc la possibilité
de former des pores en se refroidissant.\\

Il s'agit d'un effet indésirable du processus de remise en solution, d'autant plus que 
les deux traitements de revenu successifs ne parvienne pas à supprimer ces pores. 
Cependant, ce n'est pas un problème important car les bénéfices apportés par la diminution des 
agrégats eutectiques sont supérieurs, notamment en termes de résistance au fluage. \\

Nous obtenons les résultats suivants à l'aide d'ImageJ :

\begin{table}[H]
    \centering
    \caption{Proportion des différents défauts dans l'échantillon après remise en solution}
    \begin{tabular}{c|c}
        \textbf{Type de défaut}  & \textbf{Proportion observée}  \\
        \hline
        Pore               & 0,525  \% \\
        Agrégat eutectique & 0,0041 \% \\
    \end{tabular}

\end{table}

Ces mesures corroborent nos observations visuelles.\\

Nous pouvons donc produire le graphique de comparaison suivant :


\begin{figure}[htbp]
    \centering
    % This file was created by tikzplotlib v0.9.8.
\begin{tikzpicture}

\definecolor{color0}{rgb}{0.12156862745098,0.466666666666667,0.705882352941177}
\definecolor{color1}{rgb}{1,0.498039215686275,0.0549019607843137}

\begin{axis}[
legend cell align={left},
legend style={
  fill opacity=0.8,
  draw opacity=1,
  text opacity=1,
  at={(0.03,0.97)},
  anchor=north west,
  draw=white!80!black
},
tick align=outside,
tick pos=left,
title={Evolution des défauts dans les échantillons
selon les traitements thermiques appliqués},
x grid style={white!69.0196078431373!black},
xlabel={Etape de traitement thermique de l'échantillon},
xmin=-0.23, xmax=1.53,
xtick style={color=black},
xtick={0.15,1.15},
xticklabels={Brut,Remis en solution},
y grid style={white!69.0196078431373!black},
ylabel={Proportion surfacique du type de défaut},
ymin=0, ymax=0.55125,
ytick style={color=black},
ytick={0,0.1,0.2,0.3,0.4,0.5,0.6},
yticklabels={0.00\%,0.10\%,0.20\%,0.30\%,0.40\%,0.50\%,0.60\%}
]
\draw[draw=none,fill=color0,fill opacity=0.9] (axis cs:-0.15,0) rectangle (axis cs:0.15,0.34);
\addlegendimage{ybar,ybar legend,draw=none,fill=color0,fill opacity=0.9};
\addlegendentry{Pores}

\draw[draw=none,fill=color0,fill opacity=0.9] (axis cs:0.85,0) rectangle (axis cs:1.15,0.525);
\draw[draw=none,fill=color1,fill opacity=0.9] (axis cs:0.15,0) rectangle (axis cs:0.45,0.252);
\addlegendimage{ybar,ybar legend,draw=none,fill=color1,fill opacity=0.9};
\addlegendentry{Agrégats eutectiques}

\draw[draw=none,fill=color1,fill opacity=0.9] (axis cs:1.15,0) rectangle (axis cs:1.45,0.0041);
\end{axis}

\end{tikzpicture}

    \caption{Évolution des défauts dans les échantillons selon les traitements thermiques appliqués}
    \label{<label>}
\end{figure}



\section*{Conclusion scientifique}
Les différentes observations au microscope optique et au MEB ont révélé le comportement des différentes 
phases, des dendrites et des pores face aux différents traitements thermiques. 
\\
\\
Nous avons pu constater au MEB que les micro-structures sont effectivement différentes suivant les différents 
traitements appliqués. Nous avons en particulier estimé les tailles de précipités ainsi que leur fraction 
surfacique. En comparaison avec l’état initial (brut), les précipités après traitements thermique semblent 
plus organisés et alignés sur les clichés MEB. Il apparaît également une relative diminution de la part 
d'agrégats eutectiques et ce dès le premier traitement thermique.
\\
\\

En revanche nous observons que les hétérogénéité chimiques persistent partiellement après le traitement. Il 
existe d'une part des agrégats eutectiques et des dendrites résiduels. En effet, lors de la remise en 
solution, es atomes plus lourd qui diffusent plus lentement n’ont pas le temps de se répartir de manière 
homogène. Cette hétérogénéité de répartition des éléments constitue un défaut persistant qui affecte la durée 
de vie en fluage du matériau. 
\\
\\

Durant la solidification, le flux de soluté n’est parfois pas suffisant dans les zones inter-dendritiques 
laissant place à des vides appelées pores ou retassures. Pendant une sollicitation thermo-mécanique, un champ 
de contraintes est alors présent autour des pores, pouvant aller jusqu’à l’amorçage d’une fissure. Les pores 
constituent donc un défaut sérieux de la matière. Nous avons observé que le traitement thermique ne réduit en 
rien la proportion de pores. Au contraire cette part augmente puisque pendant la remise en solution, les 
atomes légers des phases liquides diffusent avec un flux plus important que les atomes lourds dans l'autre 
sens. Cette différence de flux est à l'origine de la croissance des pores. A défaut de pouvoir diminuer leur 
part, il convient de vérifier que la proportion de pores reste sous un seuil acceptable à l'issue du 
traitement.
\\
\\


Les deux dernières étapes du traitement thermique, à savoir le premier et deuxième revenu permettent une 
structure plus quadrillée de la phase $\gamma$ dans la phase $\gamma'$. Elles permettent également de faire 
croître la phase $\gamma'$ pour obtenir une taille optimale de précipité. Théoriquement il serait possible 
d'obtenir cette taille optimale directement après la remise en solution en jouant sur la vitesse de 
refroidissement mais en pratique, compte tenu de la géométrie de l'aube, il n'est pas possible d'avoir un 
gradient de température uniforme. D'où l'intérêt des deux étapes de revenu dans le traitement thermique.
\\


\section*{Conclusion générale}
Nous avons trouvé très enrichissante l'opportunité de rencontrer des doctorants au cours de cette journée.
Leur manière d'expliquer et d'appréhender la mécanique des matériaux est bien différente de celle des 
enseignants et nous avons tous senti que cette double approche a été très bénéfique pour notre apprentissage 
de la matière. Nous sommes aussi conscients de la chance que nous avons eu de passer une journée entière avec 
des chercheurs qui étaient heureux de répondre à toutes nos questions et de nous partager leurs activités de 
recherche ainsi que de nous accompagner tant sur la partie théorique qu'expérimentale. 

Cette journée a été intense intellectuellement et nous avons appris à appréhender un sujet complexe, 
découvert en peu de temps. Nous saluons pour cela les qualités de pédagogie de nos encadrants qui 
ont su nous faire digérer les concepts de manière progressive tout en nous faisant percevoir une 
logique globale. L'approche expérimentale du problème que nous avons pu mener était un élément clé pour notre compréhension puisque voir les choses de ses propres yeux aide beaucoup à l'apprentissage.

Enfin, c'était pour nous trois notre première expérience dans un laboratoire de recherche 
industrielle. Nous avons été impressionnés par les équipements que nous avons été amenés à utiliser, notamment le MEB. 

Merci à tous les scientifiques impliqués dans l'enseignement de Matériaux pour l'Ingénieur !






\end{document}