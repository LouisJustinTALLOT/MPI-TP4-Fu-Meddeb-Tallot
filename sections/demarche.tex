1. Préparation métallographique des échantillons : polissage.\\
2. Observations des microstructures au microscope optique.\\
3. Attaque chimique à l’eau régale.\\
4. Observation des microstructures au microscope électronique à balayage (MEB)\\
5. Analyses d’images : à partir des images obtenues au microscope optique (au moins 5 champs d’observation par échantillon), déterminer les fractions de défauts observés (agrégats
eutectiques ($\gamma / \gamma'$), pores et autres).\\
6. Etude de la ségrégation chimique des éléments en fonction du traitement thermique par
profils de composition chimique par analyse dispersive en énergie (EDS).\\
7. Analyse des résultats :\\
a. Conclusion sur les types de défauts observés à l’état brut de fonderie. Donner une explication quant à l’apparition de ces défauts.\\
b. Tracer les courbes d’évolution des fractions de défauts en fonction des traitements thermiques reçus.\\
c. Conclure quant à l’influence des traitements thermiques sur la fraction de défauts et sur la microstructure $\gamma / \gamma'$. \\
\newpage