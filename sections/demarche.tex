
\begin{enumerate}
    \item Préparation métallographique des échantillons : polissage.
    \item Observations des microstructures au microscope optique.
    \item Attaque chimique à l’eau régale.
    \item Observation des microstructures au microscope électronique à balayage (MEB)
    \item Analyses d’images : à partir des images obtenues au microscope optique (au moins 5 champs d’observation par échantillon), déterminer les fractions de défauts observés (agrégats
    eutectiques ($\gamma / \gamma'$), pores et autres).
    \item Etude de la ségrégation chimique des éléments en fonction du traitement thermique par
    profils de composition chimique par analyse dispersive en énergie (EDS).
    \item Analyse des résultats :
        \begin{enumerate}
            \item Conclusion sur les types de défauts observés à l’état brut de fonderie. Donner une explication quant à l’apparition de ces défauts.
            \item Tracer les courbes d’évolution des fractions de défauts en fonction des traitements thermiques reçus.
            \item Conclure quant à l’influence des traitements thermiques sur la fraction de défauts et sur la microstructure $\gamma / \gamma'$. 
        \end{enumerate}
\end{enumerate}

