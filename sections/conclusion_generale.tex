Nous avons trouvé très enrichissante l'opportunité de rencontrer des doctorants au cours de cette journée. Leur manière d'expliquer et d'appréhender la mécanique des matériaux est bien différente de celle des enseignants et nous avons tous senti que cette double approche a été très bénéfique pour notre apprentissage de la matière. Nous sommes aussi conscients de la chance que nous avons eu de passer une journée entière avec des chercheurs qui étaient heureux de répondre à toutes nos questions et de nous partager leur domaine de recherche en nous accompagnant tant sur la partie théorique qu'expérimentale. 

Cette journée a été très intense intellectuellement et cela a été très satisfaisant de pouvoir, à la fin de la journée, appréhender un sujet complexe sur lequel on ne connaissait encore rien le matin. Nous saluons pour cela la très grande pédagogie de nos encadrants qui ont su nous faire digérer les concepts de manière progressive tout en nous faisant percevoir une logique globale que nous ne quittions jamais. L'approche expérimentale du problème que nous avons pu mener était également précieuse pour notre compréhension puisque voir les choses de ses propres yeux aide beaucoup à l'apprentissage.

Enfin, c'était pour nous trois notre première expérience dans un laboratoire de recherche. Nous avons été impressionnés par les équipements que nous avons été amenés à utiliser, notamment le MEB. Pour ma part j'avais fait mon TPE sur la cristallographie et la diffraction aux rayons X et je n'avais pu voir un tel dispositif à l'époque.

En bref cette journée était bien supérieure à une journée de cours, tant au niveau de l'enrichissement personnel que de l'apprentissage scientifique.