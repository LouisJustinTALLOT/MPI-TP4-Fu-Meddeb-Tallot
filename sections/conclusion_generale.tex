Nous avons trouvé très enrichissante l'opportunité de rencontrer des doctorants au cours de cette journée.
Leur manière d'expliquer et d'appréhender la mécanique des matériaux est bien différente de celle des 
enseignants et nous avons tous senti que cette double approche a été très bénéfique pour notre apprentissage 
de la matière. Nous sommes aussi conscients de la chance que nous avons eu de passer une journée entière avec 
des chercheurs qui étaient heureux de répondre à toutes nos questions et de nous partager leurs activités de 
recherche ainsi que de nous accompagner tant sur la partie théorique qu'expérimentale. 

Cette journée a été intense intellectuellement et nous avons appris à appréhender un sujet complexe, 
découvert en peu de temps. Nous saluons pour cela les qualités de pédagogie de nos encadrants qui 
ont su nous faire digérer les concepts de manière progressive tout en nous faisant percevoir une 
logique globale. L'approche expérimentale du problème que nous avons pu mener était un élément clé pour notre compréhension puisque voir les choses de ses propres yeux aide beaucoup à l'apprentissage.

Enfin, c'était pour nous trois notre première expérience dans un laboratoire de recherche 
industrielle. Nous avons été impressionnés par les équipements que nous avons été amenés à utiliser, notamment le MEB. 

Merci à tous les scientifiques impliqués dans l'enseignement de Matériaux pour l'Ingénieur !