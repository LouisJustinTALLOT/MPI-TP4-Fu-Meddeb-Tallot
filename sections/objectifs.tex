Pour pouvoir obtenir la microstructure composée des phases $\gamma$ et $\gamma'$
avec des taillex et fractions de précipités optimales, les superalliages à base de
nickel sont soumis à une gamme de traitements thermiques : remise en solution puis
trempe, premier revenu et deuxième revenu. L’objectif de la remise en solution
(à des températures correspondantes au domaine monophasé $\gamma$) est de faire
disparaître les défauts qui sont apparus pendant le procédé de fonderie : 
ségrégations chimiques et zones d’agrégats eutectiques ($\gamma / \gamma'$).\\

Pendant la trempe (pour revenir à une température correspondant au domaine
$\gamma + \gamma'$), la microstructure issue de la remise en solution est figée 
et il n’y aurait pas de transformation de phase. Les premier et deuxième revenus 
(à des températures correspondantes au domaine biphasé $\gamma / \gamma'$) conduisent
à la précipitation et croissance de la phase $\gamma'$ dans la matrice $\gamma$. 
Les propriétés finales des superalliages à base de nickel dépendent de la microstructure 
finale en termes de taille et de fraction de précipités $\gamma'$ ainsi que de la 
fraction de défauts issues du procédé de fonderie qui n’ont pas disparu pendant 
les traitements thermiques.\\


L’objectif de ce travail pratique est d’étudier l’influence des traitements thermiques 
sur la microstructure et sur les défauts formés pendant le procédé de fonderie : pores, 
zones d’agrégats eutectiques et hétérogénéités chimiques entre dendrites et zones 
inter-dendritiques. Pour cela, des observations de microstructures couplées avec de 
l’analyse d’image seront réalisées.
