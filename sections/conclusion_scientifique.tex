Les différentes observations au microscope optique et au MEB ont révélé le comportement des différentes phases, des dendrites et des pores face aux différents traitements thermiques. 
\\
\\
Nous avons pu constater au MEB que les micro-structures sont effectivement différentes suivant les différents traitements appliqués. Nous avons en particulier estimé les tailles de précipités ainsi que leur fraction surfacique. En comparaison avec l’état initial (brut), les précipités après traitements thermique semblent plus organisés et alignés sur les clichés MEB. Il apparaît également une diminution de la part d'agrégats eutectiques et ce dès le premier traitement thermique.
\\
\\
Un des défauts résultant du procédé de fonderie, et qui a déjà été évoqué, est la présence d’une hétérogénéité chimique due à la croissance dendritique qui persiste malgré l’application de traitements thermiques. Les atomes plus lourd (Ta, W, Re) qui diffusent plus lentement n’ont pas le temps durant la remise en solution de se répartir de manière homogène. Des analyses EPMA sur un super-alliage après traitement thermique montrent la disparité de composition qu’il est possible de retrouver dans les zones dendritiques et interdendritiques.
Cette ségrégation résiduelle induit une précipitation légèrement différente dans chacune des zones, qui va modifier le désaccord paramétrique naturel, qui dépend de la composition de chacune des phases. Ceci influence le comportement du matériau, notamment en fluage où une densité de dislocations plus importante a été observée dans le cœur des dendrites.
Durant la solidification, le flux de soluté n’est parfois pas assez suffisant dans les zones interdendritiques laissent place à des vides appelées pores ou retassures. En général, ils sont de forme arrondie et distants de l'ordre de l'espace interdendritique secondaire (150 à 550 $\mu$m) avec une dimension maximale comprise entre 40 $\mu$m et 60 $\mu$m pour un rayon moyen de 5$\mu$m . Lorsque les zones interdendritiques sont contigües ou interconnectées, des cavités plus grandes peuvent être observées. Pendant une sollicitation thermomécanique, un champ de concentration de contraintes est alors présent autour des pores, pouvant aller jusqu’à l’amorçage d’une fissure.