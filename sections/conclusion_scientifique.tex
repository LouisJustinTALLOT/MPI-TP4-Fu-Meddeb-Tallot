Les différentes observations au microscope optique et au MEB ont révélé le comportement des
différentes phases, des dendrites et des pores face aux traitements thermiques.\\


Nous avons pu constater au MEB que les micro-structures évoluent suivant les  
traitements appliqués. Nous avons en particulier estimé les tailles de précipités ainsi que leur fraction 
surfacique. En comparaison avec l’état initial (brut), les précipités après traitements thermiques semblent 
plus organisés et alignés sur les clichés MEB. Il apparaît également une diminution de la part 
d'agrégats eutectiques et ce dès le premier traitement thermique.\\



En revanche nous observons que les hétérogénéités chimiques persistent partiellement après le traitement. Il 
existe des agrégats eutectiques et des dendrites résiduels. En effet, lors de la remise en 
solution, les atomes plus lourd qui diffusent plus lentement n’ont pas le temps de se répartir de manière 
homogène. Cette hétérogénéité de répartition des éléments constitue un défaut persistant qui affecte la durée 
de vie en fluage du matériau.\\



Durant la solidification, le flux de soluté n’est parfois pas suffisant dans les zones inter-dendritiques 
laissant place à des vides appelées pores ou retassures. Pendant une sollicitation thermo-mécanique, un champ 
de contraintes est alors présent autour des pores, pouvant aller jusqu’à l’amorçage d’une fissure. Les pores 
constituent donc un défaut sérieux de la matière. Nous avons observé que le traitement thermique ne réduit en 
rien la proportion de pores. Au contraire, cette part augmente puisque pendant la remise en solution, les 
atomes légers des phases liquides diffusent avec un flux plus important que les atomes lourds dans l'autre 
sens. Cette différence de flux est à l'origine de la croissance des pores. A défaut de pouvoir diminuer leur 
part, il convient de vérifier que la proportion de pores reste sous un seuil acceptable à l'issue du 
traitement.\\



Les deux dernières étapes du traitement thermique, à savoir le premier et deuxième revenu permettent une 
structure plus "quadrillée" de la phase $\gamma$ dans la phase $\gamma'$. Elles permettent également de faire 
croître la phase $\gamma'$ pour obtenir une taille optimale de précipité. Théoriquement, il serait possible 
d'obtenir celle-ci directement après la remise en solution en jouant sur la vitesse de 
refroidissement ; cependant en pratique, compte-tenu de la géométrie de l'aube, il n'est pas possible d'avoir un gradient de température uniforme, d'où l'intérêt des deux étapes de revenu dans le traitement thermique.\\
