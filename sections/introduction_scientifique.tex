L'aube de turbine évolue dans un environnement particulièrement oxydant et corrosif. En particulier la haute température (1250°C) et les contraintes auxquelles est soumise l'aube sont propices au fluage : la déformation lente de l'aube soumise à une contrainte constante sous une forte température. Il convient donc d'étudier l'influence des caractéristiques micro-structurales sur le fluage, dont la diffusion est le principal vecteur puisque c'est sous son effet que les atomes migrent provoquant l'allongement du matériau.

La première forte évolution technologique dans le domaine a été le passage à la conception d'aube mono-cristallines vers la fin des années 1970. En effet les joints de grain sont des "autoroutes à diffusion" et le fait de n'avoir qu'un seul grain améliore grandement les performances en fluage. Ainsi les aubes sont aujourd'hui réalisées en super-alliage de Nickel à base d'Aluminium par refroidissement et croissance d'un unique grain initial.

D'autre part les défauts micro-structurels ont également un impact sur la résistance mécanique du matériau. Par exemple la trop forte présence de pores pose problème ainsi que les hétérogénéités chimiques. Lors de l'étape de fonderie, le refroidissement n'est pas homogène et certaines zones se solidifient avant d'autres ce qui fait que le liquide est appauvri de certains éléments : on parle alors de dendrites. C'est ce contraste chimique résultant entre dendrites et zones inter-dendritiques qui pose problème au niveau de la structure.

À l'échelle des phases, un agencement ordonnée des phases $\gamma'$ dans la matrice $\gamma$ permet de moins laisser se propager les champs de contraintes de certains défauts tel que la dislocation.
\\
\\

Deux questions se posent : Quelles sont les caractéristiques micro-structurelles optimales du point de vue de la résistance du matériau au fluage et aux autres contraintes de son milieu ? Et dans quelle mesure différents traitements thermiques permettent-ils d'atteindre cet optimum ?

La première question est traitée par les industriels qui ont déterminé différentes caractéristiques reconnues comme optimum (fraction volumique de phase $\gamma'$, taille des précipités de phase $\gamma'$, différence des paramètres de maille entre $\gamma$ et $\gamma'$...).
Nous nous concentrons donc dans ce TP sur l'influence du traitement thermique en observant l'organisation et les défauts de la micro-structure après chaque étape de ce traitement.

\newpage